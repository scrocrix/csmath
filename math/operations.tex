\documentclass{article}
\usepackage[utf8]{inputenc}

\usepackage{amsthm}
\usepackage{amsmath}
\usepackage{algorithm2e}
\usepackage{listings}

\newtheorem{theorem}{Theorem}
\newtheorem{lemma}[theorem]{Lemma}

\begin{document}

\section{operations.ComplexMultiply}

Prove the correctness of method that multiply two natural numbers, for all integer constants $c \geq 2$.

\begin{theorem}
The algorithm grows at a $O(2^n)$ complexity.

\begin{lemma}
The algorithm can be mathematically written as follows:

\[
    f(y,z) =
    \begin{cases}
        0, \text{if } z = 0\\
        f(c \times y, \frac{z}{c}) + y \times  z \bmod{c}, \text{otherwise}
    \end{cases}
\]

\begin{itemize}
  \item So that, $y, z \geq 0$
  \item And, $f(y, 0) = 0 \mid y \in Z$
  \item And, $c \geq 2$
\end{itemize}

\begin{proof}

We can prove the formula by means of induction.

Basis mode: $y = 1, z = 1, c = 2$
\begin{align*}
    f(1, 1) = f(2 \times 1, \frac{1}{2}) + 1 \times 1 \bmod{2} = f(2, \frac{1}{2})
\end{align*}

Now we have to find the result value of $f(2, \frac{1}{2})$. The machine will simplify $\frac{1}{2}$ into $0.5$ which is fraction of zero, so:
\begin{align*}
    f(2, 0.5) = 0 \text{, because } f(y, 0) = 0 \mid y \in Z
\end{align*}

Induction mode: $y = 10, z = 10, c = 2$

\begin{align*}
    f(10, 10) = f(2 \times 10, \frac{10}{2}) + 10 \times 10 \bmod{2} = f(20, 5) + 10 \times 10 \bmod{2}
\end{align*}

Now let's find what's the value of $f(20, 5)$
\begin{align*}
    f(20, 5) = f(2 \times 20, \frac{5}{2}) + 20 \times 5 \bmod{2} = f(40, \frac{5}{2}) + 20 \times 5 \bmod{2}
\end{align*}

Now let's find what's the value of $f(40, \frac{5}{2})$
\begin{align*}
    f(40, \frac{5}{2}) = f(2 \times 40, \frac{\frac{5}{2}}{2}) + 40 \times \frac{5}{2} \bmod{2} = f(80, \frac{5}{4}) + 50 \times \frac{5}{2} \bmod{2}
\end{align*}

Now let's find what's the value of $f(80, \frac{5}{4})$
\begin{align*}
    f(80, \frac{5}{4}) = f(2 \times 80, \frac{\frac{5}{4}}{2}) + 80 \times \frac{5}{4} \bmod{2} = f(160, \frac{5}{8}) + 80 \times \frac{5}{4} \bmod{2}
\end{align*}

Finally. I'll skip ahead and affirm that $f(160, \frac{5}{8}) = 0$.

\end{proof}

\end{lemma}

\end{theorem}

\end{document}
